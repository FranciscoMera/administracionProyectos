\documentclass[11pt]{book}

\usepackage{admin/bookTheme}

\title{$\Pi$ Card}
\author{\textbf{Equipo 5}\\Caballero Ramírez Michelle\\Campos Tenorio Diana Yesenia\\Gamboam Meléndez Cristian Neftali\\García Paul Alberto\\Mera Torres Francisco Isidoro}
\subtitle{Plan de negocios}
\lider{Alberto García Paul}
\aprueba{Hernandez Jaime Josefina}
\date{}

\begin{document}
	\ThisLRCornerWallPaper{1}{img/imgFooter/footPortada}
	\maketitle
	\LRCornerWallPaper{1}{img/imgFooter/footPages}
	\makeProjectCharter
	\makeFirmas
	\tableofcontents
%\part{Actividad 6}	

\chapter{Descripción del Proyecto}
%Secciones

\section{Descripción del problema}

	\subsection{Necesidad del producto}

	\subsection{Modelo de negocio}

	\subsection{Estadística}

	\subsection{Beneficio Social y Sustentabilidad}

	\subsection{Sector económico}

\chapter{Estructura de Mercado y Marketing}
%Secciones

\section{Investigación del Mercado}

	\subsection{Segmentación del Mercado}
	
	\subsection{Determinación y Proyección de la Demanda}
	
	\subsection{Competidores}
	
\section{Estrategias de Mercadotecnia}

	\subsection{Diseño de la imagen}
	
	\subsection{Precio de venta}
	
	\subsection{Plaza}
	
	\subsection{Promoción y Publicidad}


\chapter{Estructura Operacional}
\section{Ingeniería del proyecto}
	
	\subsection{Proceso de Producción}
	
	\subsection{Diagrama de Producción}
	
	\subsection{Requerimientos de maquinaria, equipo y mano de obra}

\section{Distribución de Instalaciones}

	\subsection{Áreas y Departamentos de la Empresa}
	
	\subsection{Distribución de Instalaciones}

\section{Localización de Instalaciones}

	\subsection{Macrolocalización}
	
	\subsection{Microlocalización}

\section{Conclusiones de la Estructura Operacional}


\chapter{Estructura Estratégica y Organizacional}
\section{Elementos Estratégicos}

	\subsection{Nombre, slogan, giro}
	
	\subsection{Misión, Visión, Valores y Políticas}
	
	\subsection{Objetivos}
	
	\subsection{FODA}
	
\section{Conclusiones del Estudio Administrativo}



\chapter{Estructura Económico-Financiera}
\section{Presupuestos}


\section{Estados financieros proforma}


\section{Análisis financiero}

\section{Evaluación del Proyecto}


\section{Desición Sobre el Proyecto}

%\chapter{Etapa de Exploración}
	
%	\section{Identificar una necesidad.}

%	Implementación de un módulo interconectable con dispositivos que permite realizar transacciones bancarias, consulta de información personal y localización.
	
%	\subsection{Segmentación}

%	Se han idenfiticado los siguientes grupos:
%	\begin{itemize}
	%	\item \textbf{Hombres y mujeres de 25 a 50 años:} Tienen una mayor posibilidad de tener diferentes tarjetas debido a que en su mayoria laboran.
	%	\item \textbf{Personas de ente 16 a 24 años:} Son estudiantes y laboran que cuenten con 
	%	\item \textbf{Personas de más de 50 años:} Estas personas pueden llegar a tener más de una tarjeta.
	%\end{itemize}
	
	%Se tomará al primer grupo de segmentación porque en su mayoria son personas que ya laboran y son cuentahabientes en las diferentes institucion bancarias del México actual(2017).
		
	%A personas mayores de 16 años que sean cuentahabientes de diferentes sucursales bancarias.
	
	%\subsection{Productos Similares}
	%\begin{TablaGenerica}{Productos similares}
		%\tableItem{Moneda localizadora}{Localizar objetos}{Ninguna}{\$30.00 dlls}{\begin{Titemize}
			%	\Titem Pequeño
			%	\Titem Discreto
			%	\Titem Pegable
			%	\Titem Pila de 7 horas
			%	\Titem Inalámbrico
			%	\Titem GPS
			%\end{Titemize}}{-}{No alcanzó objetivo de financiamento}
	%\tableItem{TrackR Bravo}{Localizar un objeto}{TrackR USA}{\$669.05 MXN}{\begin{Titemize}
		%	\Titem GPS
		%	\Titem Bluetooth
		%	\Titem Utiliza sonidos
		%	\Titem Se puede colocar en casi cualquier objeto
		%	\Titem Si te encuentras lejos del objeto que perdiste se puede localizar por medio de la red.
		%\end{Titemize}}{-}{-}
	%\tableItem{TrackR Wallet 2.0}{Localizar una cartera}{TrackR USA}{\$35.0 dlls}{\begin{Titemize}
		%	\Titem Delgado
		%	\Titem Utiliza una app para localizar a la cartera
		%\end{Titemize}}{-}{-}
	%\tableItem{Tile}{Localiza objetos}{Tile}{\$30.00 dlls.}{\begin{Titemize}
		%	\Titem Delgado.
		%	\Titem Bluetooth
		%\end{Titemize}}{-}{-}
	%\end{TablaGenerica}


%\part{Actividad 7}

%\chapter{¿Qué quiere o necesita mi cliente?}

%\section{Introducción breve}

%Somos estudiantes universitarios del IPN. Nos encontramos investigando el manejo de datos personales durante las transacciones bancarias utilizando tarjetas de débito y crédito, mediante una encuesta\footnote{ver sección \ref{sec:Encuesta}} que tiene como propósito identificar las necesidades y problemáticas en el uso de las tarjetas que tienen los clientes bancarios.

%\section{Encuesta}
%\label{sec:Encuesta}

%\subsection{Ficha de identificación}

%\begin{longtable}{| p{0.35\textwidth}| p{0.35\textwidth}| p {0.20\textwidth}|}
%	\hline
%	\rowcolor{sectionChapter}
%	\multicolumn{3}{c}{\bf\color{white}Ficha de identificación}\\
%	\hline
%	\endhead
%	\bf Género: & \bf Edad: & \bf Estado civil:\\
%	\hrule & \hrule &  \hrule\\
%	& &\\
%	\bf Ocupación: & & \\
%	\hrule & \hrule & \hrule\\
%	\hline
%\end{longtable}

%\subsection{Preguntas}
%\begin{enumerate}
%	\item ¿Cuentas con tarjeta de crédito o débito? Si sí:
%		\begin{enumerate}
%			\item ¿Se encuentran asociadas en más de una institución bancaria?
%			\item ¿Te interesaría una tarjeta universal en la que pudieras manejar todas
%			tus cuentas?
%		\end{enumerate}
%	\item ¿Has conocido a alguien que haya perdido su tarjeta o tu lo has vivido?Si sí:
%		\begin{enumerate}
%			\item ¿Te gustaría que la tarjeta que te presentamos fuera localizable?
%		\end{enumerate}
%	\item ¿Has escuchado el término \textbf{clonación de tarjeta}?
		
%	\item ¿Te gustaría que por medio de una aplicación dieras la autorización para realizar movimientos con tus cuentas bancarias?
	
%	\item ¿Qué información personal te gustaría guardar en esta tarjeta?
	
%	\item ¿Te gustaría definir un límite en la tarjeta para evitar usar más dinero del que dispones o cuentas?
	
%	\item Después de haber contestado estas preguntas ¿ Te interesaría adquirir un producto con estas características? Si sí:
%		\begin{enumerate}
%			\item ¿A qué precio lo adquirirías?
%			\item ¿Te gustaría que los bancos te mencionarán acerca de esta tarjeta?
%		\end{enumerate}
%\end{enumerate}

%\chapter{Conclusiones}
\end{document}